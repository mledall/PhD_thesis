\documentclass[12pt,oneside]{book}
\include{macros/style}
%\include{macros/use_packages}

% math packages ------
\usepackage{amsmath,amsfonts,amssymb,cancel,bm}
\usepackage[fixamsmath]{mathtools}
\usepackage[vcentermath]{youngtab}				% Source http://www.ctex.org/documents/packages/math/youngtab.pdf

% Graphics packages -------
\usepackage{graphicx,epstopdf,subfig}

% Useful packages ------
\usepackage{comment, hyperref}
\usepackage[colorinlistoftodos]{todonotes}
\usepackage{diagbox}

\newcommand{\nc}{\newcommand}

% Environment commands --------------
\def\beq{\begin{equation}}
\def\eeq{\end{equation}}
\def\beqs{\begin{equation*}}
\def\eeqs{\end{equation*}}
\def\bsubeq{\begin{subequations}}
\def\esubeq{\end{subequations}}
\def\bpm{\begin{pmatrix}}
\def\epm{\end{pmatrix}}
\def\bit{\begin{itemize}}
\def\eit{\end{itemize}}
\def\ben{\begin{enumerate}}
\def\een{\end{enumerate}}
\def\btab{\begin{tabular}}
\def\etab{\end{tabular}}
% Greek letters ---------------
\nc{\al}{\alpha}
\nc{\ga}{\gamma}
\nc{\de}{\delta}
\nc{\ep}{\epsilon}
\nc{\ze}{\zeta}
\nc{\et}{\eta}
\nc{\ka}{\kappa}
\nc{\la}{\lambda}
\nc{\rh}{\rho}
\nc{\si}{\sigma}
\nc{\ta}{\tau}
\nc{\up}{\upsilon}
\nc{\ph}{\phi}
\nc{\ch}{\chi}
\nc{\ps}{\psi}
\nc{\om}{\omega}
\nc{\Ga}{\Gamma}
\nc{\De}{\Delta}
\nc{\La}{\Lambda}
\nc{\Si}{\Sigma}
\nc{\Up}{\Upsilon}
\nc{\Ph}{\Phi}
\nc{\Ps}{\Psi}
\nc{\Om}{\Omega}

% Math/physics commands ----------
\nc{\sket}[1]{|#1\rangle}
\nc{\sbra}[1]{\langle#1|}
\nc{\ket}[1]{\left|#1\right\rangle}
\nc{\bra}[1]{\left\langle#1\right|}
\nc{\vev}[1]{\left\langle#1\right\rangle}
\nc{\svev}[1]{\langle#1\rangle}
\nc{\TO}[1]{\vev{\mathcal{T}#1}}
\nc{\scalarp}[2]{\left\langle#1|#2\right\rangle}
\nc{\sscalarp}[2]{\langle#1|#2\rangle}
\nc{\commut}[2]{\left[#1,#2\right]}
\nc{\anticommut}[2]{\left\{#1,#2\right\}}
\nc{\scommut}[2]{[#1,#2]}
\nc{\santicommut}[2]{\{#1,#2\}}
\nc{\Leq}{\leqslant}
\nc{\Geq}{\geqslant}
\nc{\unit}{1\!\!1}
\nc{\bk}{\mathbf k}
\nc{\MM}{\mathcal{M}}
\nc{\Int}{\int\frac{d^4l}{(2\pi)^4}}
\nc{\ev}{\text{eV}}
\nc{\gev}{\text{GeV}}
\nc{\tev}{\text{TeV}}
\nc{\ptl}{\partial}
\nc{\ov}{\overline}
\nc{\Neq}[1]{n^{eq}_{#1}}
\nc{\Yeq}[1]{Y^{eq}_{#1}}
\nc{\gammaeq}[1]{\gamma^{eq}_{#1}}
\nc{\CPV}{\mathcal P}
\nc{\bdim}[1]{\underline{\bm{#1}}}

% commenting commands ------
\nc{\MLD}[1]{\color{red} [MLD:#1] \color{black}}


\begin{document}

% Front Matter
\input frontmatter/fm

\newpage


	\include{chapters/prologue/prologue}

\part{Particle Physics}

	\include{chapters/PP/intro/PPintro}
	\include{chapters/PP/background/PPbackground}
	\include{chapters/PP/HHS_Lepto/HHS_Lepto}
	\include{chapters/PP/neutrinos/low_mass_neutrinos}
	\include{chapters/PP/conclusion/PPconclusion}
%	\appendix
%	\include{chapters/PP/appendix/PPapp}
	
\part{Condensed Matter}

	\include{chapters/CM/intro/CMintro}
	\include{chapters/CM/background/CMbackground}
	\include{chapters/CM/CMimpurity_statesSC/CMimpurity_statesSC}
	\include{chapters/CM/spectro/CMspectroscopy}
%	\include{chapters/CM/sum_rules/CMsum_rules}
	\include{chapters/CM/conclusion/CMconclusion}
	
%	\include{chapters/prologue/epilogue}
	
%	\appendix
%	\include{chapters/CM/appendix/CMapp}

	\TOCadd{Bibliography}
	\bibliographystyle{phaip} %"plain" style organizes reference in alphabetical order. phaip
	\bibliography{Thesis_bib}

	%\bibliographystyle{phaip}% Phys Rev Bibtex Style: apsrev ; American Institute of Physics Journal : phaip
	
\end{document}
